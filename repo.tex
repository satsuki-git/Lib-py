% ctrl-alt-b でビルド
% ここからヘッダ部
\documentclass[a4paper,10pt,titlepage]{jsarticle}
%使用パッケージ設定
\usepackage[margin=20mm,includefoot]{geometry}
\usepackage[dvipdfmx]{graphicx}
\usepackage{url}
\usepackage{moreverb}
\usepackage{framed}
\usepackage{amsmath}
\usepackage{bm}
\usepackage{here}
\usepackage{amssymb}
\usepackage{listings}
\usepackage{ascmac}
\usepackage{listings,jlisting}
\usepackage{ulem}
\usepackage{comment}
%\usepackage[dvipdfmx]{hyperref}
%\usepackage{pxjahyper}
% 自作命令設定
\newcommand{\ttt}[1]{\texttt{#1}}
%ここからソースコードの表示に関する設定
\lstset{
  basicstyle={\ttfamily},
  identifierstyle={\small},
  commentstyle={\smallitshape},
  keywordstyle={\small\bfseries},
  ndkeywordstyle={\small},
  stringstyle={\small\ttfamily},
  frame={tb},
  breaklines=true,
  columns=[l]{fullflexible},
  numbers=left,
  xrightmargin=0zw,
  xleftmargin=3zw,
  numberstyle={\scriptsize},
  stepnumber=1,
  numbersep=1zw,
  lineskip=-0.5ex
}
%ここまでソースコードの表示に関する設定
% maketitle用の設定
\title{辞書的な何か(python)}
\date{更新日:2020-07-09}
\author{作成者:Me}

%頁番号の有無
%\pagestyle{empty}
% ここから本文
\begin{document}
% タイトルページの作成
\maketitle

\section{A}

\section{B}

\section{C}

\section{D}

\section{E}

\section{F}

\textbf{format}:文字列内に変数を埋め込むときに使用するメソッド.\\
参照:\url{https://www.sejuku.net/blog/22481}

\section{G}

\section{H}

\section{I}

\section{J}

\section{K}

\textbf{Keras}:Pythonで書かれた,TensorFlowまたはCNTK,Theano上で実行可能な高水準のニューラルネットワークライブラリ.様々なモジュールがある.\quad $Sequential,Conv2d$他多数.参照:\url{https://qiita.com/sasayabaku/items/9e376ba8e38efe3bcf79}

\section{L}

\section{M}

\section{N}

\section{O}

\textbf{OS}:OSに依存しているさまざまな機能を利用するためのモジュール.\\
主にファイルやディレクトリ操作が可能で、ファイルの一覧やpathを取得できたり、新規にファイル・ディレクトリを作成することができる.参照:\url{https://www.sejuku.net/blog/67787}


\section{P}

\section{Q}

\section{R}

\section{S}

\textbf{SHUTIL}:ファイルやフォルダ(ディレクトリ)を簡易に扱うためのモジュール.\\
osモジュールより高水準.ファイル,フォルダなどをコピーできる.参照:\url{https://techacademy.jp/magazine/31470}

\section{T}

\section{U}

\section{V}

\section{W}

\section{X}

\section{Y}

\section{Z}


\begin{comment}
\begin{itemize}
  \item
\end{itemize}


\begin{figure}[H]
 \begin{center}
  \includegraphics[width=90mm]{LDI.png}
 \end{center}
 \caption{Layered Depth Image}
 \label{fig:LDI}
\end{figure}

\begin{thebibliography}{99}
  \bibitem{url:LDI} \url{https://www.art-science.org/journal/v3n1/v3n1pp008/artsci-v3n1pp008.pdf}
\end{thebibliography}
\end{comment}


\end{document}
